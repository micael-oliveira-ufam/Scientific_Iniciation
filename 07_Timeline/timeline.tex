
% code by Andrew:
% http://tex.stackexchange.com/a/28452/13304
\makeatletter
\let\matamp=&
\catcode`\&=13
\makeatletter
\def&{\iftikz@is@matrix
  \pgfmatrixnextcell
  \else
  \matamp
  \fi}
\makeatother

\newcounter{lines}
\def\endlr{\stepcounter{lines}\\}

\newcounter{vtml}
\setcounter{vtml}{0}

\newif\ifvtimelinetitle
\newif\ifvtimebottomline
\tikzset{description/.style={
  column 2/.append style={#1}
 },
 timeline color/.store in=\vtmlcolor,
 timeline color=red!80!black,
 timeline color st/.style={fill=\vtmlcolor,draw=\vtmlcolor},
 use timeline header/.is if=vtimelinetitle,
 use timeline header=false,
 add bottom line/.is if=vtimebottomline,
 add bottom line=false,
 timeline title/.store in=\vtimelinetitle,
 timeline title={},
 line offset/.store in=\lineoffset,
 line offset=6pt,
}

\NewEnviron{vtimeline}[1][]{%
\setcounter{lines}{1}%
\stepcounter{vtml}%
\begin{tikzpicture}[column 1/.style={anchor=east},
 column 2/.style={anchor=west},
 text depth=0pt,text height=1ex,
 row sep=1ex,
 column sep=1em,
 #1
]
\matrix(vtimeline\thevtml)[matrix of nodes]{\BODY};
\pgfmathtruncatemacro\endmtx{\thelines-1}
\path[timeline color st] 
($(vtimeline\thevtml-1-1.north east)!0.5!(vtimeline\thevtml-1-2.north west)$)--
($(vtimeline\thevtml-\endmtx-1.south east)!0.5!(vtimeline\thevtml-\endmtx-2.south west)$);
\foreach \x in {1,...,\endmtx}{
 \node[circle,timeline color st, inner sep=0.15pt, draw=white, thick] 
 (vtimeline\thevtml-c-\x) at 
 ($(vtimeline\thevtml-\x-1.east)!0.5!(vtimeline\thevtml-\x-2.west)$){};
 \draw[timeline color st](vtimeline\thevtml-c-\x.west)--++(-3pt,0);
 }
 \ifvtimelinetitle%
  \draw[timeline color st]([yshift=\lineoffset]vtimeline\thevtml.north west)--
  ([yshift=\lineoffset]vtimeline\thevtml.north east);
  \node[anchor=west,yshift=16pt,font=\large]
   at (vtimeline\thevtml-1-1.north west) 
   {\textsc{Timeline \thevtml}: \textit{\vtimelinetitle}};
 \else%
  \relax%
 \fi%
 \ifvtimebottomline%
   \draw[timeline color st]([yshift=-\lineoffset]vtimeline\thevtml.south west)--
  ([yshift=-\lineoffset]vtimeline\thevtml.south east);
 \else%
   \relax%
 \fi%
\end{tikzpicture}
}

\footnotesize
\justify
\begin{vtimeline}[description={text width=8.4cm},
 row sep=16ex, 
 timeline color=cyan!80!blue,
 timeline title={Cronograma proposto ao andamento da pesquisa.}]


 
\textcolor{blue!30!green!75}{Agosto(2020)} $\rightarrow$ \textcolor{gray}{Julho(2021)} & Levantamento bibliográfico sobre as hipóteses centrais do Transtorno Depressivo. Avanços na literatura entre as ferramentas de otimização, triagem virtual, ADME e docking molecular.  \endlr

\textcolor{gray}{Agosto(2020)} $\rightarrow$ \textcolor{gray}{Setembro(2020)}   & Triagem virtual baseada na similaridade espacial com os inibidores consolidados de \textit{IL-6R}. Seleções dos inibidores que obedecem à Regra de Lipinski. Otimização e estudo dos descritores moleculares. \endlr

\textcolor{gray}{Outubro(2020)} $\rightarrow$ \textcolor{gray}{Janeiro(2020)}  & Preparação do receptor \textit{IL-6R} mediante protonação por adição de Hidrogênio e cargas parciais. Estudos de docking do tipo semirrígido-flexível nos softwares AutoDock Vina e GOLD.\endlr

\textcolor{gray}{Fevereiro(2021)} $\rightarrow$ \textcolor{gray}{Abril(2021)}  & Revisão e possíveis correções de todos os cálculos efetuados e análise comparativa com pesquisas já consolidadas na literatura. \endlr

\textcolor{gray}{Maio(2021)} $\rightarrow$ \textcolor{gray}{Junho(2021)} &  Análise das conclusões, estudo dos mecanismos de síntese dos fármacos mais promissores e perspectivas futuras. \endlr

\textcolor{gray}{Junho(2021)} $\rightarrow$ \textcolor{gray}{Julho(2021)} & Predição da probabilidade de interações com outras proteínas no organismo. Considerações éticas sobre o tratamento psicofarmacológico. \endlr

\textcolor{gray}{Julho(2021)} & Elaboração do resumo e relatório final de Iniciação Científica. \textcolor{red!80}{(atividade obrigatória)}  \endlr

\textcolor{gray}{Julho(2021)} & Preparação da apresentação final para o Congresso de Iniciação Científica. \textcolor{red!80}{(atividade obrigatória)} \endlr
\end{vtimeline}
