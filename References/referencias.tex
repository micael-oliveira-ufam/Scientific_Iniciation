\begin{thebibliography}{00}

    \bibitem{1} Charney, Dennis S.; Neurobiology of mental illness - 4th edition. Oxford University Press, 2013. p.496
    
    \bibitem{2} Wong, M.L., \& Licinio, J. (2004). From monoamines to genomic targets: A paradigm shift for drug discovery in depression. \textbf{Nature Reviews Drug Discovery}, Vol.3, pp. 136-151.
    
    \bibitem{3} Tacchi, Mary Jane; \& Scott, Jan; Depression: A very short introduction - 1th edition. Oxford University Press, 2017. p.91
    
    \bibitem{4} Nemeroff, C.B., \& Vale, W. W. (2005). The neurobiology of depression: inroads to treatment and new drug discovery. \textbf{Neurobiology of Depression}, Vol. 66.
    
    \bibitem{5}
    
    \bibitem{6} Stahl, Stephen M.; Psicofarmacologia: bases neurocientíficas e aplicações práticas - 4ª edição. GEN, 2014. p.
    
    \bibitem{7} Jeon, S. W., \& Kim, Y. K. (2016). Molecular neurobiology and promising new treatment in depression. \textbf{International Journal of Molecular Sciences}, 17(3).
    
    \bibitem{8} Krishnan, V., \& Nestler, E. J. (2008). The molecular neurobiology of depression. \textbf{Nature}, Vol. 455, pp.894-902.
    
    \bibitem{9} Yanik, M., Erel; \& Erel, O.; \& Kati, M. (2004). The relationship between potency of oxidative stress and severity of depression. \textbf{Acta Neuropsychiatrica}, 16(4), 200-203.
    
    \bibitem{10} Schatzberg, Alan F. Manual de psicofarmacologia clínica. - 6ª edição. Artmed, 2009. p.58
    
    \bibitem{11} Ciraulo, Domenic A. [et al.]; Pharmacotherapy of Depression - 2th edition. Humana Press, 2011. p.
    
    \bibitem{12} Belmaker, R.H.; \& Agam, Galila. (2008). \textbf{The New England Journal of Medicine.} Vol.358, p.55-68.   
    
    \bibitem{13} Richards, C. Steven; O'Hara, Michael W. (2014). The Oxford Handbook of Depression and Comorbidity. Oxford University Press. p. .
    
    \bibitem{14} American Psychiatric Association (2013), Diagnostic and Statistical Manual of Mental Disorders - 5th edition. Arlington: American Psychiatric Publishing, pp. 165.
    
    \bibitem{15} Stein, Dan J.; \& Kupfer, David J. [et al.]. Textbook of mood disorders. (2006). American Psychiatric Publishing - 1st edition. p.38
    
    \bibitem{16} Kiametis, Alessandra Sofia. Modelagem molecular de potenciais candidatos a inibidores da acetilcolinesterase. 76 f., il. Tese(Doutorado em Física) — Universidade de Brasília, Brasília, 2012.
    
    \bibitem{17} Silva, Mônica de Abreu. Modelos preditivos baseados em descritores moleculares e modos de interação receptor-ligante para inibidores de Acetilcolinesterase. 143 f., il. Tese(Doutorado em Física) - Universidade de Brasília, Brasília, 2017.
    
    \bibitem{18} Berton, O. \& Nestler, E.J. New approaches to antidepressant drug discovery: beyond monoamines. \textbf{Nature Reviews Neuroscience.} v.7, pp.137-151 (2006). 
    
    \bibitem{19} Nestler, E.J. (et al.) Treatment resistant depression: A multi-scale, systems biology approach.
                 \textbf{Neuroscience & Biobehavioral Reviews}, v.84, pp.272-288 (2018).
                 
    \bibitem{20} Bissete, Garth [et al.] Elevated Concentrations of CRF in the Locus Coeruleus of Depressed Subjects. \textbf{Nature Neuropsychopharmacology}, v.28, pp.1328-1335 (2003).
    
    \bibitem{21} Ramachandran, K.I. [et al.] Computational Chemistry and Molecular Modeling. Springer Publishing, (2008), pp.37
    
    \bibitem{22} Durrant, J.D. \& McCammon, J.A. Molecular dynamics simulations and drug discovery. \textbf{BMC Biology}, v.9, pp.71 (2011).
    
    \bibitem{23} Kishimoto, T. [et al.]. Single-dose infusion ketamine and non-ketamine N-methyl-d-aspartate receptor antagonists for unipolar and bipolar depression: A meta-analysis of efficacy, safety and time trajectories. \textbf{Psychological Medicine}, v.46 (7), pp.1459-1472 (2016)
    
\end{thebibliography}